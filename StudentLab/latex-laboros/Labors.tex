\documentclass[a4paper,12pt]{article}
\usepackage{build/sagatave}

\id{je11011}
\author{Jānis Erdmanis}
\kurss{Fizikas Bakalaura 2. kurss}
\numurs{9}
\title{Elektrona īpatnējais lādiņš}


\newcommand{\plot}[2]{
	\begin{figure}[!ht]
		\caption{#2}
		\center
		\includegraphics[width=0.8\textwidth]{#1}
	\end{figure}
}


\newcommand{\subplot}[2]{
	\subfigure[#2]{
%            \label{fig:first}
%			width=0.5\textwidth
            \includegraphics[width=0.35\textwidth]{#1}
           	}%
}



\begin{document}
\maketitle

\section*{Darba mērķi}

\begin{itemize}


	
	\item teksts
	
	
	\item tek
\end{itemize}

\section*{Darba uzdevumi}

\section*{Darba piederumi}
	
\section*{Mērijumi un to apstrāde}

\begin{table}[ht!]
	\caption{Mana tabula}
	\center
	\small
	\csvautotabular{build/dati.csv}
\end{table}
	

	\begin{figure}[ht!] % Ātrumu sadalījums
		\caption{Virsraksraksts}	
		\center
		
			\subplot{build/.1graf.png}{1. grafiks}
			\subplot{build/.2graf.png}{2. grafiks}\\	
	\end{figure}
	
	\plot{build/.1graf.png}{Cits grafiks}
	
\clearpage
\pagebreak

\section*{Rezultāti}

\section*{Secinājumi}


\end{document}
% =================================================== 
