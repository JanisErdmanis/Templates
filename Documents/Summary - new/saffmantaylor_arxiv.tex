\documentclass[aip,preprint,jmp,showpacs,showkeys]{revtex4-1}

%\documentclass[aip,reprint]{revtex4-1}
\usepackage[scanall]{psfrag}
\usepackage{graphicx}% Include figure files
\usepackage{amsmath}
%\usepackage{dcolumn}% Align table columns on decimal point
\usepackage{bm}% bold math

\usepackage{overpic}

\usepackage{inputenc}
\usepackage{fontspec}

\begin{document}

% Use the \preprint command to place your local institutional report number 
% on the title page in preprint mode.
% Multiple \preprint commands are allowed.

\preprint{}

\title{A nēw prediction of wavelength selection in radial viscous fingering involving normal and tangential stresses.}

\author{Mathias \surname{Nagel}}
% \altaffiliation{Laboratory of Fluid Mechanics and Instabilities}%Lines break automatically or can be forced with \\
  \email{mathias.nagel@epfl.ch}
\author{Fran\c{c}ois \surname{Gallaire}}%
\affiliation{ Laboratory of Fluid Mechanics and Instabilities - EPFL Lausanne, Switzerland%\\This line break forced with \textbackslash\textbackslash
}%

% repeat the \author .. \affiliation  etc. as needed
% \email, \thanks, \homepage, \altaffiliation all apply to the current author.
% Explanatory text should go in the []'s, 
% actual e-mail address or url should go in the {}'s for \email and \homepage.
% Please use the appropriate macro for the type of information

% \affiliation command applies to all authors since the last \affiliation command. 
% The \affiliation command should follow the other information.

%\author{}
%\email[]{Your e-mail address}
%\homepage[]{Your web page}
%\thanks{}
%\altaffiliation{}
%\affiliation{}

% Collaboration name, if desired (requires use of superscriptaddress option in \documentclass). 
% \noaffiliation is required (may also be used with the \author command).
%\collaboration{}
%\noaffiliation
\newcommand{\order}[1]{$\mathcal{O}(#1)$}

\date{\today}

\begin{abstract}

We reconsider the radial Saffman-Taylor instability, when a fluid injected from a point source displaces another fluid with a higher viscosity in a Hele-Shaw cell, where the fluids are confined between two neighboring flat plates. The advancing fluid front is unstable and forms fingers along the circumference.
The so-called Brinkman equations is used to describe the flow field, which also takes into account viscous stresses in the plane and not only viscous stresses due to the confining plates like the Darcy equation. The dispersion relation agrees better with the experimental results than the classical linear stability analysis of radial fingering in Hele-Shaw cells that uses Darcy's law as a model for the fluid motion. 

\end{abstract}



\pacs{47.15.gp}% insert suggested PACS numbers in braces on next line
\keywords{Saffman-Taylor instability, Hele-Shaw cell, Brinkman equation}

\maketitle %\maketitle must follow title, authors, abstract and \pacs


\section{Introduction}

Viscous fingering, also called Saffman-Taylor\cite{SaffmanTaylor58} instability, is considered as an archetype of pattern forming instability (see Couder 2000\cite{couder2000} for an insightful review). It has also been widely studied in the context of industrial research, such as petroleum extraction in particular. The phenomenon belongs to the broad family of instabilities of growth in Laplacian fields, which includes solidification, aggregation, etc... It was first studied by Saffman and Taylor\cite{SaffmanTaylor58} in 1958, who observed the formation of patterns upon injection of a fluid into a channel filled with another more viscous fluid. Saffman and Taylor\cite{SaffmanTaylor58} studied the formation of fingers in shallow rectangular channels and observed that the formation of fingers was dependent on the ratio of channel height to channel width. The extension to radial geometry, depicted in figure \ref{fig:isopressure}a, dates back to Bataille\cite{Bataille68}, Wilson\cite{Wilson75} and Paterson\cite{Paterson81}.


\normalsize
%\bibliography{nagelbib1}
\begin{thebibliography}{}

%\renewcommand {\baselinestretch}{0.8}\small \normalsize



\bibitem{SaffmanTaylor58}
{Saffman, P. G., Taylor, G. I.} 1958
{The Penetration of a Fluid into a Porous Medium or Hele-Shaw Cell Containing a More Viscous Liquid.}
\textit{Proc. Roy. Soc. A}
\textbf{245} (1242) 312-329

\bibitem{couder2000}
{Couder Y.} 2000
{Viscous fingering as an archetype of growth patterns}
\textit{In Perspectives in fluid dynamics, edited by Batchelor, G. K.,  Moffatt H. K. and  Worster, M. G. }
53-104

\bibitem{Bataille68}
{Bataille, J.} 1968
{Stabilit\'e d'un déplacement radial non miscible}
\textit{Revue Inst. P\'etrole}
\textbf{23} 1349-1364

\bibitem{Wilson75}
{Wilson, S.D.R} 1975
{A note on the measurement of dynamic contact angles.}
\textit{Journal of Colloid and Interface Science}
\textbf{51}(3) 
%532 - 534

\bibitem{Paterson81}
{Paterson, L.} 1981
{Radial fingering in a Hele Shaw cell.}
\textit{J. Fluid Mech.}
\textbf{113} 513-529

\bibitem{Maxworthy89}
{Maxworthy, T.} 1989
{Experimental study of interface instability in a Hele-Shaw cell.}
\textit{Phys. Rev. A},
\textbf{39} (11) 5863-5866

\bibitem{Chuoke}
{Chuoke, R. L., van Meurs, P. and van der Pol, C.} 1959
{The instability of slow immiscible viscous liquid-liquid  displacements in permeable media}
\textit{Petrol. Trans. AIME}
\textbf{216} 188-194

\bibitem{Shelley93}
{Dai, W.-S., Shelley, M. J}
{A numerical study of the effect of surface tension and noise on an expanding Hele-Shaw bubble}
\textit{Phys. Fluids}
\textbf{5}(9) 2131-2146

\bibitem{Paterson84}
{Paterson, L.} 1985
{Fingering with miscible fluids in a Hele Shaw cell.}
\textit{Phys. Fluids},
\textbf{28} (1) 26-30

\bibitem{KimHomsy09}
{Kim, H., Funada, T., Joseph, D. D., Homsy, G. M.}
{Viscous potential flow analysis of radial fingering in a Hele-Shaw cell.}
\textit{Phys. Fluids}
\textbf{21} 074106

\bibitem{boudaoud}
{Ben Amar M. and Boudaoud A.} 2009
Suction in Darcy and Stokes interfacial flows: Maximum growth rate versus minimum dissipation.
\textit{Eur. Phys. J. Special Topics} 
\textbf{166} 83–88

%\bibitem{Joseph05}
%{Jospeh, D.D.} 2006
%{Potential flow of viscous fluids: Historical notes}
%\textit{Int. J. Multiphase Flow}
%\textbf{32} 285-310
 
\bibitem{Smirnov10}
{Logvinov, O. A., Ivashnyov, O. E., Smirnov, N. N.} 2010
{Evaluation of viscous fingering width in Hele-Shaw flows.}
\textit{Acta Astronautica}
\textbf{67} 53-59

%\bibitem{pozrikidis92}
%{Pozrikidis, C.} 1992
%{Boundary-Integral and Singularity Methods for Linearized Viscous Flow.}
%\textit{Cambridge U.P., Cambridge}

%\bibitem{Tryggvason83}
%{Tryggvason, G., Aref, H.} 1983
%{Numerical experiments on Hele Shaw flow with a sharp interface.}
%\textit{J. Fluid Mech.}
%\textbf{136} 1-30

\bibitem{Fernandez}
{J. Fernandez, J., Kurowski P., Limat L. and Petitjeans P.} 2001
{Wavelength selection of fingering instability inside Hele-Shaw cells}
\textit{Phys. Fluids}
\textbf{13} 3120-3125

\bibitem{Carles}
{Carl\`es P., Huang Z., Carbone G. and Rosenblatt C.} 2006
{Rayleigh-Taylor Instability for Immiscible Fluids of Arbitrary Viscosities: A Magnetic Levitation Investigation and Theoretical Model}
\textit{Phys. Rev. Lett.}
\textbf{96} 104501


\bibitem{Boos}
{Boos W. and Thess, A.} 1997
{Thermocapillary flow in a {Hele-Shaw} cell}
\textit{J. Fluid Mech.}
\textbf{352} 305-320

\bibitem{Gallaire}
{Gallaire F., Meliga P., Laure P. and Baroud, C.} 2013
{Marangoni-induced force on a drop in a Hele-Shaw cell}
\textit{submitted}


\bibitem{ParkHomsy84a}
{Park, C.-W., Homsy, G. M.} 1984
{Two-phase displacement in Hele Shaw cells: theory.}
\textit{J. Fluid Mech.}
\textbf{139} 291-308

\end{thebibliography}

\end{document}